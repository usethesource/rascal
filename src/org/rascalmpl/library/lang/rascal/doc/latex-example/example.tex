\RequirePackage[l2tabu, orthodox]{nag} // check common latex errors
\documentclass[]{article}

\usepackage{fixltx2e} // fix latex2e errors
\usepackage[T1]{fontenc}
\usepackage{microtype}
\usepackage{lmodern}

% real rascal highligh specific stuff
\usepackage{highlight}
\usepackage{rascal}
\usepackage{listings}

\begin{document}
	You can have inline rascal \irascal{data X = y(int x);} or you can have a rascal
	code listing:

	\begin{rascal}
		data X = answer(int fortyTwo);
	\end{rascal}
			
	If you want to list a whole module, then use the rascalModule environment as
	below.

	\begin{rascalModule}
		module NewModule

		@doc{Example docs}
		data X = answer(int fortyTwo);

		public X getAnswer() = answer(42);
	\end{rascalModule}
	
\end{document}
